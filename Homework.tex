\documentclass[12pt]{article}
	
\usepackage[margin=1in]{geometry}		% For setting margins
\usepackage{amsmath}				% For Math
\usepackage{fancyhdr}				% For fancy header/footer
\usepackage{graphicx}				% For including figure/image
\usepackage{cancel}					% To use the slash to cancel out stuff in work

%%%%%%%%%%%%%%%%%%%%%%
% Set up fancy header/footer
\pagestyle{fancy}
\fancyhead[LO,L]{Amir H. Hazini (218245696)}
\fancyhead[CO,C]{PHYS2040 - Homework 2-3}
\fancyhead[RO,R]{\today}
\fancyfoot[LO,L]{}
\fancyfoot[CO,C]{\thepage}
\fancyfoot[RO,R]{}
\renewcommand{\headrulewidth}{0.4pt}
\renewcommand{\footrulewidth}{0.4pt}
%%%%%%%%%%%%%%%%%%%%%%

\begin{document}
%%\noindent \textit{A uniformly charged dielectric gel having charge density $\rho_v = \rho_o$ and dielectric constant of $\epsilon_r = 2$ is enclosed inside a dielectric shell with dielectric constant of $\epsilon_r = 5$ as shown in the figure. The dielectric shell is surrounded by free space ($\epsilon_r = 1$). Determine E, $\Phi$, D, P in all regions.}

%%\begin{figure}[!h] 
%%	\begin{centering}
		%%\includegraphics[keepaspectratio = true, width = 2in]{image.PNG}
		%%\caption{Given scenario.}
%%	\end{centering}
%%\end{figure}

\section{Galilean and Lorentz Transformation}
\noindent \textbf{Question 1.a)} Describe what happens to the angle, $\alpha = \tan^{-1}\left(\frac{v}{c}\right)$, and, thus, the transformed axes in Fig. 5.17 in the text, as the relative velocity of the frames $S$ and $S'$ approaches zero.\\
\\
\noindent \textbf{Question 1.b)} What happens as the speed $v$ approaches the speed of light?\\
\\
\noindent \textbf{Question 1.c)} Provide sketches on graph paper of the situations $\beta = \beta_1 = \frac{1}{4}$, $\beta = \beta_2 = \frac{1}{2}$, $\beta = \beta_3 = \frac{3}{4}$.\\
\\
\noindent \textbf{Question 2)} Galilean relativity: A woman standing still at a train station watches two girls throwing a tennis ball in a moving train. Suppose the train is moving east with a constant speed of $25 \, \text{m/s}$, and one of the girls throws the ball with a speed of $10 \, \text{m/s}$ with respect to herself towards the other girl, who is $5 \, \text{m}$ west from her. What is the velocity of the ball as observed by the woman on the station? Explain your reasoning.\\
\\
\noindent \textbf{Question 3)} In a frame $S$, two events are observed by Max: event 1: a pion ($\pi$ meson) is created at rest at the origin, and event 2: the pion disintegrates (decays) after time $\tau$. Susan is an observer in frame $S'$, and is moving in the positive direction along the positive $x$-axis, with speed $v$. Susan observes the same two events in her frame. The two frames coincide at time $t = t' = 0$.\\
\\
\noindent \textbf{Question 3.a)} Find the positions and timings of these two events in Susan’s frame according to the Galilean transformation.\\
\\
\noindent \textbf{Question 3.b)} What are the positions and timings of the two events in Susan’s frame according to the Lorentz transformation?\\
\\
Provide details of how you calculated your results, and explain the differences between the answers (a) and (b).\\
\\
\underline{Electric Flux Density:}
	\begin{align*}
		{x}'=\gamma (x-vt)\\
		{t}'=\gamma (x-vt)\\
		{\gamma}= 
		\oint_S \vec{D} \cdot \hat{n} &= Q_{en} \\
		\int_0^\pi \int_0^{2\pi} D_r r^2 sin\theta d\phi d\theta &= Q_{en}\\	
		4\pi r^2 D_r &= Q_{en}
	\end{align*}

\section{Lorentz Transformation: Velocity Addition}
\noindent \textbf{Question 1)} A space ship flies past earth at 0.90c. As it goes by it fires a bullet in the forward
direction at 0.95c with respect to the rocket. What is the bullet’s speed with respect
to earth?\\
\\
\noindent \textbf{Question 2)} Now look at a related problem: a space ship flying past earth at 0.90c. As it goes
by it fires a laser. Use the relativistic velocity addition to show at what speed the
laser beam is recorded by an earth observer.\\
\\
\noindent \textbf{Question 3)} What is the relative velocity of two space ships if one fires a missile at the other
at 0.85c and the other observes it to approach at 0.99c?\\
\\

	
Final Answer Summary:
		
\begin{tabular}{| c | c | c | c |}
\hline
& Region 1 ($r<a$)& Region 2 ($a\leq r < b$)& Region 3 ($r\geq b$)\\ \hline
$\vec{D}~(C/m^2)$ & $\frac{\rho_o r}{3} \hat{r}$ & $\frac{\rho_o a^3}{3 r^2} \hat{r}$ & $\frac{\rho_o a^3}{3 r^2} \hat{r}$\\
$\vec{E}~(V/m)$ & $\frac{\rho_o r}{6 \epsilon_o} \hat{r}$ & $\frac{1}{5 \epsilon_o} \frac{\rho_o a^3}{3 r^2} \hat{r}$ & $\frac{1}{\epsilon_o} \frac{\rho_o a^3}{3 r^2} \hat{r}$\\
$\vec{P}~(C/m^2)$ & $\frac{\rho_o r}{6} \hat{r}$ & $\frac{4\rho_o a^3}{15 r^2} \hat{r}$ & $0 \hat{r}$\\
$\Phi~(V)$ &  $\frac{\rho_o a^2}{\epsilon_o}\left[ \frac{a}{3b} + \frac{1}{15} + \frac{1}{12}\right] - \frac{\rho_o}{\epsilon_o}\left[\frac{a^3}{15b} + \frac{r^2}{12}\right]$ & $\frac{\rho_o a^3}{3 \epsilon_o b} + \frac{\rho_o a^3}{15 \epsilon_o r} - \frac{\rho_o a^3}{15 \epsilon_o b}$ & $\frac{\rho_o a^3}{3\epsilon_o r}$\\ \hline
\end{tabular}

\end{document}